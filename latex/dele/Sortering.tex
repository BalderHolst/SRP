\chapter{Sorteringsalgoritmer}
\label{ch:Sorteringsalgoritmer}

\section{Hvad er sortering?}
\label{sec:Hvad er sortering?}

\begin{figure}
	\begin{center}
		$$[4,2,5,3,1] \:\:\longrightarrow\:\: [1,2,3,4,5]$$
	\end{center}
	\caption{Eksempel på sortering af en liste}
	\label{fig:Eksempel på sortering af en liste}
\end{figure}


Sortering er helt lavpraktisk at sætte en mængde data en rækkefølge på baggrund af dataens attributter. Det kunne f.eks. være alfabetisk eller efter farve eller størrelse. Det er dog ikke entydigt hvilken sorterings-strategi der ville være hurtigst. Disse sorterings-strategier kan også kaldes sorterings-algoritmer, det er ikke kun mennesker der kan benytte sorterings-algoritmer til f.eks. at sortere kort, men computere kan også, og de gør det for det meste også hurtigere. De næste to afsnit omhandler to forskellige sorterings-algoritmer, og deres måde at sortere en liste med tal.\\


Herfra vil jeg kun forholde mig til sortering af lister med tal, og sortere dem på baggrund af deres størrelse (se figur \ref{fig:Eksempel på sortering af en liste})


\section{Insertionsort}
\label{sec:Insertionsort}

Insertionsort er en af de mere simple sorteringsalgoritmer. Dette er pseudokode for algoritmen \cite[s. 104]{aogd}.

\begin{figure}
	\begin{center}

		\begin{lstlisting}
		funktion insertionsort(liste) {
			for i = 1 til i = n {					# her er n længden af listen
			e = liste[i]	 						# dette er elementet i listen

			if e < liste[0] {						# hvis elementet er større end det første element i listen
			for j = j til j = 0{ 
			liste[j] = liste[j-1]		# ryk elementerne på pladserne 0 til j, et tak frem
			}
			liste[0] = e 						# sætter dette element forrest i listen
			}
			else{
				j = i 								# j er en nu tæller der starter på i
				while liste[j-1] > e {			# kør mens liste[j-1] er størren end elementet
				liste[j] = liste[j - 1] # ryk liste[j] et tak til højre
				j -= 1
				}
				l[j] = element						# indsæt elementet hvor det passer ind
			}
			}			
			return(liste)								# returnerer den sorterede liste
		}

		\end{lstlisting}
	\end{center}
	\caption{Pseudokode til insertionsort}
	\label{fig:Pseudokode til insertionsort}
\end{figure}


Det er ikke en tilfældighed at denne algoritme hedder insertionsort. Den fungerer nemlig ved at gennemgå gennemgå hvert element i listen, og placere det hvor det passer ind i de elementer der allerede er sorterede. Dette er nok til dels den måde man f.eks. ville sortere sin hånd i Uno eller 500. At algoritmen gennemgår alle elementerne i listen kronologisk kan vi se allerede i linje 2, da algoritmen her begynder med en for-lykke, der tæller for hvert element i listen, dog starter den ved 2. element i listen. For at gøre koden mere læsbar sættes elementet som algoritmen er nået til ind i variablen e i linje 3. Det næste algoritmen gør, er at checke om elementet har en mindre værdi end det første element i listen, hvis dette er sandt rykkes rykkes alt før elementet et tak til højre, og elementet sættes ind først i listen se figur \ref{fig:Indsæt element først i listen}

\blue{balder}

\begin{figure}
	\begin{center}
		$$[\blue{5},\blue{7},\blue{9},\red{4},10,6,2]$$
		$$[\blue{5},\blue{5},\blue{7},\blue{9},10,6,2]$$
		$$[\red{4},\blue{5},\blue{7},\blue{9},10,6,2]$$
	\end{center}
	\caption{Indsæt element først i listen}
	\label{fig:Indsæt element først i listen}
\end{figure}


(se afsnit \ref{sec:Analyse af Insertionsort} for store-O analysen)

\section{Mergesort}
\label{sec:Mergesort}

%\lstinputlisting[language=Python]{../python/algoritmer/mergesort.py}

\section{Analyse af Insertionsort}
\label{sec:Analyse af Insertionsort}


