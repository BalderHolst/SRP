\chapter{Algoritmers Udførelsestid}
\label{ch:Algoritmers Udførelsestid}

\section{Tid som Funktion af Argumentets Kardinalitet}
\label{sec:Tid som Funktion af Argumentets Kardinalitet}

Dette afsnit bygger primært på side $24$ og $25$ i bogen \emph{"Algoritmer og Datastrukturer"} \cite{aogd}.\\

Når vi analyserer algoritmer, er det primære formål at skabe udsagn, der kort og præcist beskriver algoritmers opførsel. I vores tilfælde er vi interesserede i algoritmens udførselsestid. Vi lader $\I _n$ være mængden af mulige argumenter til algoritmen, $n$ være størrelsen af $\I _n$ og $I$ være en instans af $\I$ (altså $I \in \I$). Vi kan 

\begin{figure}
	\begin{center}
		\padtable
		\begin{tabular}{l}
			\hline
			$O(f(n)) = {g(n): \exists c > 0: \exists n_0 \in \N_+: \forall n \geq n_0: g(n) \leq c \cdot f(n)}$\\
			$\Omega (f(n)) = {g(n): \exists c > 0: \exists n_0 \in \N_+: \forall n \geq n_0: g(n) \geq c \cdot f(n)}$\\
			$\Theta (f(n)) = O(f(n)) \cup \Omega (f(n))$\\
			\hline
		\end{tabular}
	\end{center}
	%\vspace{-3mm}
	\caption{Definition af $O(f(x))$, $\Omega (f(x))$ og $\Theta (f(x))$ \cite[s. 26]{aogd}.}
	\label{fig:Store-O definition}
\end{figure}


